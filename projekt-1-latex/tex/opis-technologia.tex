\section{Opis Technologii Połączenia, Identyfikacji i Transmisji Danych}

\subsection{Technologia Połączenia}
\begin{itemize}
    \item \textbf{GSM (Global System for Mobile Communications)}: Jest to technologia komórkowa wykorzystywana do połączeń głosowych oraz przesyłania danych. Operator Plus wykorzystuje technologię GSM w swoich starszych sieciach, w obszarach, gdzie technologia 3G i LTE nie jest jeszcze dostępna. Technologia GSM stanowi podstawę dla wielu usług mobilnych, umożliwiając klientom nawiązywanie połączeń głosowych i korzystanie z podstawowych usług danych, takich jak wiadomości tekstowe.

    \item \textbf{3G (Third Generation)}: Operator Plus oferuje także technologię 3G, która umożliwia przesyłanie danych z wyższymi prędkościami niż GSM. Dzięki technologii 3G klienci mogą korzystać z usług internetowych, przeglądać strony internetowe i korzystać z aplikacji mobilnych.

    \item \textbf{LTE (Long-Term Evolution)}: Operator Plus dostarcza usługi w technologii LTE, co pozwala na jeszcze szybszą transmisję danych. Sieci LTE pozwalają klientom na strumieniowanie wideo w jakości HD, granie online i inne wymagające aplikacje, które wymagają dużej przepustowości.

\end{itemize}

\subsection{Identyfikacja}
\begin{itemize}
    \item \textbf{Karty SIM}: Klienci operatora Plus otrzymują karty SIM (Subscriber Identity Module), które odgrywają kluczową rolę w identyfikacji użytkowników w sieci. Karty SIM zawierają informacje identyfikacyjne klienta, takie jak numer telefonu i identyfikator sieci. Karty SIM pozwalają klientom korzystać z usług operatora, a także przenosić swoje informacje i numer telefonu między różnymi urządzeniami.

\end{itemize}

\subsection{Transmisja Danych}
\begin{itemize}
    \item \textbf{Technologia Pakietowa}: Plus zapewnia transmisję danych z wykorzystaniem technologii pakietowej. Dane są przesyłane w formie pakietów, co pozwala na efektywne wykorzystanie dostępnej przepustowości sieci. Dzięki technologii pakietowej klienci mogą przesyłać dane w sposób bardziej efektywny i optymalny. Jest to szczególnie istotne w przypadku korzystania z usług internetowych, przeglądania stron, a także korzystania z aplikacji, które wymagają szybkiej transmisji danych.

\end{itemize}
