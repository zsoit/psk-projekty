\section{Obsługiwane Protokoły Sieciowe przez Operatora Plus}

Operator Plus obsługuje wiele protokołów sieciowych, które umożliwiają klientom pełny dostęp do różnych usług internetowych. Poniżej znajduje się lista obsługiwanych protokołów:

\begin{itemize}
    \item \textbf{Protokół IP (Internet Protocol)}: Ten protokół jest podstawowym narzędziem do przesyłania danych w sieci internetowej. Działa na zasadzie przekazywania paczek danych między urządzeniami w sieci. Protokół IP jest niezbędny do funkcjonowania internetu.

    \item \textbf{Protokół TCP (Transmission Control Protocol)}: TCP jest używany do zapewnienia niezawodności przesyłania danych. Zapewnia kontrolę nad łącznością między urządzeniami, gwarantując, że dane zostaną dostarczone bez utraty i w odpowiedniej kolejności.

    \item \textbf{Protokół UDP (User Datagram Protocol)}: UDP jest wykorzystywany do przesyłania danych w czasie rzeczywistym. Jest często stosowany do przesyłania strumieniowych mediów, takich jak audio i wideo, gdzie czas odpowiedzi jest ważniejszy niż dokładność.

    \item \textbf{Protokół HTTP (Hypertext Transfer Protocol)}: HTTP jest protokołem używanym do przeglądania stron internetowych. Pozwala na pobieranie zawartości stron WWW, wykonywanie żądań do serwerów i przeglądanie treści online.

    \item \textbf{Protokół FTP (File Transfer Protocol)}: FTP jest używany do przesyłania plików między komputerami. Klienci FTP mogą przesyłać, pobierać i zarządzać plikami na zdalnych serwerach. To przydatne narzędzie dla administratorów systemów i użytkowników, którzy potrzebują wymieniać pliki.

    \item \textbf{Protokół SMTP (Simple Mail Transfer Protocol)}: SMTP jest używany do przesyłania wiadomości e-mail. Ten protokół umożliwia klientom wysyłanie e-maili do serwerów pocztowych i dostarczanie ich do odpowiednich skrzynek odbiorczych.

    \item \textbf{Protokół POP3 (Post Office Protocol, version 3)}: POP3 jest protokołem służącym do pobierania wiadomości e-mail z serwera pocztowego. Klienci e-mail używają tego protokołu do pobierania swojej poczty na lokalne urządzenia.

    \item \textbf{Protokół IMAP (Internet Message Access Protocol)}: IMAP jest również stosowany w obszarze e-mail i pozwala na zarządzanie i pobieranie wiadomości z serwera pocztowego, ale w bardziej zaawansowany sposób niż POP3. Umożliwia dostęp do wiadomości bez ich pobierania.

    \item \textbf{Protokół DNS (Domain Name System)}: DNS przekształca adresy URL na adresy IP, umożliwiając klientom łatwe odnajdywanie serwerów i stron internetowych.

    \item \textbf{Protokół SSH (Secure Shell)}: SSH jest protokołem zapewniającym bezpieczny zdalny dostęp do innych komputerów. Jest używany do zdalnego zarządzania serwerami i przesyłania danych w sposób szyfrowany.
\end{itemize}
