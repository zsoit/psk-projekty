\section{Podsumowanie}
\subsection{Kluczowe Korzyści dla Firmy}

    Zastosowanie Zasilacza Bezprzerwowego (UPS):


    \begin{enumerate}

        \item Ciągłość Działania:
            \begin{itemize}
                \item Zapewnienie nieprzerwanej pracy urządzeń elektrycznych, eliminując przestoje spowodowane utratą
                zasilania.
            \end{itemize}


        \item Ochrona Elektronicznego Sprzętu:      
            \begin{itemize}
                \item Skuteczna ochrona przed szkodliwymi skokami i spadkami napięcia, minimalizująca ryzyko
                uszkodzenia sprzętu elektronicznego.
            \end{itemize}


        \item Zminimalizowanie Przerw w Pracy
            \begin{itemize}
                \item Możliwość kontynuowania pracy na sprzęcie przez pewien czas po utracie zasilania, umożliwiająca
                bezpieczne zamknięcie systemów operacyjnych i zapisanie danych.
            \end{itemize}

        \item Ochrona Przed Awariami Zasilania
            \begin{itemize}
                \item Działanie jako źródło awaryjne w przypadku awarii zasilania, co pomaga uniknąć przerw w
                funkcjonowaniu kluczowych systemów.
            \end{itemize}

        \item Zwiększenie Stabilności Systemu
            \begin{itemize}
                \item Poprawa ogólnej stabilności systemu, szczególnie w miejscach o niestabilnych warunkach zasilania.
            \end{itemize}

            \item Podtrzymywanie Zasilania w Krytycznych Aplikacjach
            \begin{itemize}
                \item Kluczowa rola w utrzymaniu zasilania systemów krytycznych, takich jak szpitale, laboratoria czy
                centra przetwarzania danych.
            \end{itemize}




         \item Podłączenie do Sieci Internet za Pomocą Łącza Satelitarnego
            \begin{itemize}
                \item Łatwa Integracja z Siecią
                \item Szybkie i efektywne połączenie z internetem, umożliwiające łatwą integrację z firmową siecią.
                Zwiększona Odporność na Warunki Pogodowe
                \item Lepsza odporność na warunki atmosferyczne, co przekłada się na stabilność łącza, nawet w trudnych
                warunkach pogodowych.
            \end{itemize}


        \item Zwiększona Odporność na Przeszkody
        
        \item Rozwiązanie VPN dla Pracowników Zdalnych:
        \begin{itemize}
            \item Bezpieczne i Szyfrowane Połączenie
            \item Zapewnienie pracownikom zdalnym bezpiecznego i szyfrowanego dostępu do firmowej sieci.
        \end{itemize}

        \item Skalowalność i Wydajność
        \begin{itemize}
            \item Możliwość obsługi dużej liczby użytkowników, co pozwala na elastyczne dostosowanie do
            zmieniających się potrzeb firmy.
        \end{itemize}

        \item Zarządzanie Tunelami VPN
        \begin{itemize}
            \item Działanie z dużą liczbą tuneli VPN, co umożliwia równoczesne bezpieczne połączenia wielu
            pracowników zdalnych.
        \end{itemize}

        \item Sprzętowy System Archiwizacji
        \begin{itemize}
            \item Skalowalność Pojemności - Możliwość skalowania do dużych pojemności, umożliwiająca efektywne przechowywanie i
            zarządzanie danymi.
            \item Wysoka Prędkość Transferu - Zapewnienie bardzo wysokiej prędkości odczytu i zapisu, co jest kluczowe dla efektywnego
            zarządzania dużymi bazami danych.
        \end{itemize}


        \item Redundancja i Bezpieczeństwo Danych
        \begin{itemize}
            \item Redundancja i Bezpieczeństwo Danych
            \item Automatyczne kopiowanie danych między różnymi lokalizacjami, co zwiększa bezpieczeństwo i
            dostępność archiwizowanych danych.
            \item Powyższe korzyści przekładają się na poprawę efektywności, bezpieczeństwa i ciągłości działania
            infrastruktury teleinformatycznej firmy.
        \end{itemize}



    \end{enumerate}


\subsection{Podsumowanie Kosztów Całkowitych}
    \begin{flushleft}
        \begin{table}[h]
            \renewcommand{\arraystretch}{1.5}
            \begin{tabular}{|l|l|}
                \hline
                \textbf{Nazwa} & \textbf{Cena (zł)} \\
                \hline
                UPS & 1 216 437,48 \\
                Antena satelitarna & 13 600,11 \\
                Abonament miesięczny za internet satelitarny & 6 051,60 \\
                VPN & 93 173,9 \\
                System Archiwizacji & 1 354 451,59 \\
                \hline
                \textbf{Suma} & \textbf{2 664 714,78} \\
                \hline
            \end{tabular}
        \end{table}
    \end{flushleft}