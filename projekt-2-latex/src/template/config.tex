\title{\mytitle}
\author{\myauthor}
\date{\mydate}


\documentclass{article}
\usepackage{amsmath}
\usepackage{subfig}

\usepackage[polish]{babel}
\usepackage[T1]{fontenc}

% pdf inlcude
\usepackage{pdfpages}
\usepackage{pgfgantt}

% font
\usepackage{helvet}
\renewcommand{\familydefault}{\sfdefault}

% margin
\usepackage[a4paper, total={7in, 8in}]{geometry}

% graphics
\usepackage{graphicx}
\graphicspath{ {./src/images/} }
\usepackage{fancyhdr}
\fancypagestyle{plain}{
    \fancyhf{}
    \renewcommand{\footrulewidth}{0pt}
    \renewcommand{\headrulewidth}{0.5pt}


    \fancyfoot[C]{\textit{\footcenter}}
    \fancyfoot[L]{\footleft}
    \fancyfoot[R]{\footrigt}
    
    \fancyhead[LE]{\textit{\leftmark}} %EVEN PAGE
    \fancyhead[RO]{\textit{\leftmark}} %odd page
}

\pagestyle{plain}

% \usepackage{hyperref}
\usepackage{color}   %May be necessary if you want to color links
\usepackage{hyperref}
\hypersetup{
    colorlinks=true, %set true if you want colored links
    linktoc=all,     %set to all if you want both sections and subsections linked
    linkcolor=blue,  %choose some color if you want links to stand out
    pdftitle={\mytitle},
    pdfsubject={\subject},
    pdfauthor={\myauthor},
    pdfkeywords={\subjectyear - \school}
}


\include{./src/template/color.tex}


