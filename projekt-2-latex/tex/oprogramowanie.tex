\section{Oprogramowanie}

\subsection{Wybór Systemów Operacyjnych}

    W ramach projektu infrastruktury sieciowej dla firmy XYZ, wybrano następujące systemy operacyjne:

    \begin{itemize}
    \item \textbf{Windows Server 2019:} System operacyjny serwera, który zapewni stabilność i niezawodność dla centralnego serwera zasobów.
    \item \textbf{Windows 10 Pro:} System operacyjny dla stacji roboczych, dostosowany do potrzeb programistów i pracowników firmy.
    \item \textbf{Linux Ubuntu:} Wykorzystany na stacjach administracyjnych, umożliwiający zarządzanie i monitorowanie sieci.
    \end{itemize}

\subsection{Oprogramowanie Użytkowe}

    W ramach infrastruktury sieciowej firmy XYZ zostaną udostępnione następujące oprogramowania użytkowe:

    \begin{itemize}
    \item \textbf{Microsoft Office 365:} Pakiet biurowy do obsługi dokumentów, komunikacji i współpracy.
    \item \textbf{Visual Studio:} Środowisko programistyczne do rozwoju aplikacji i projektów programistycznych.
    \item \textbf{Adobe Creative Cloud:} Narzędzia do projektowania grafiki i multimediów.
    \item \textbf{AutoCAD:} Oprogramowanie do projektowania CAD, przydatne w branży inżynieryjnej.
    \item \textbf{JIRA:} Narzędzie do zarządzania projektami i śledzenia zadań.
    \end{itemize}

\subsection{Narzędzia Sieciowe}

    Do zarządzania siecią i monitorowania jej wydajności, firma XYZ używać będzie następujących narzędzi sieciowych:

    \begin{itemize}
    \item \textbf{Wireshark:} Narzędzie do analizy i monitorowania ruchu sieciowego.
    \item \textbf{Nagios:} Oprogramowanie do monitorowania systemów i urządzeń sieciowych.
    \item \textbf{PuTTY:} Program do zdalnego dostępu do urządzeń i serwerów przez protokół SSH.
    \item \textbf{SolarWinds:} Narzędzie do zarządzania siecią i monitorowania jej wydajności.
    \end{itemize}
